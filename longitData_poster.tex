%%%%%%%%%%%%%%%%%%%%%%%%%%%%%%%%%%%%%%%%%
% a0poster Landscape Poster
% LaTeX Template
% Version 1.0 (22/06/13)
%
% The a0poster class was created by:
% Gerlinde Kettl and Matthias Weiser (tex@kettl.de)
% 
% This template has been downloaded from:
% http://www.LaTeXTemplates.com
%
% License:
% CC BY-NC-SA 3.0 (http://creativecommons.org/licenses/by-nc-sa/3.0/)
%
%%%%%%%%%%%%%%%%%%%%%%%%%%%%%%%%%%%%%%%%%

%----------------------------------------------------------------------------------------
%	PACKAGES AND OTHER DOCUMENT CONFIGURATIONS
%----------------------------------------------------------------------------------------

\documentclass[a0,landscape]{a0poster}

\usepackage{multicol} % This is so we can have multiple columns of text side-by-side
\columnsep=100pt % This is the amount of white space between the columns in the poster
\columnseprule=3pt % This is the thickness of the black line between the columns in the poster

\usepackage[svgnames]{xcolor} % Specify colors by their 'svgnames', for a full list of all colors available see here: http://www.latextemplates.com/svgnames-colors

\usepackage{times} % Use the times font
%\usepackage{palatino} % Uncomment to use the Palatino font

\usepackage{graphicx} % Required for including images
\graphicspath{{figures/}} % Location of the graphics files
\usepackage{booktabs} % Top and bottom rules for table
\usepackage[font=small,labelfont=bf]{caption} % Required for specifying captions to tables and figures
\usepackage{amsfonts, amsmath, amsthm, amssymb} % For math fonts, symbols and environments
\usepackage{wrapfig} % Allows wrapping text around tables and figures

\begin{document}

%----------------------------------------------------------------------------------------
%	POSTER HEADER 
%----------------------------------------------------------------------------------------

% The header is divided into three boxes:
% The first is 55% wide and houses the title, subtitle, names and university/organization
% The second is 25% wide and houses contact information
% The third is 19% wide and houses a logo for your university/organization or a photo of you
% The widths of these boxes can be easily edited to accommodate your content as you see fit

\begin{minipage}[b]{0.55\linewidth}
\veryHuge \color{NavyBlue} \textbf{Examining Tutoring Efficacy in Primary School} \color{Black}\\ % Title
\Huge\textit{...Longitudinal Modeling...}\\[1cm] % Subtitle
\huge \textbf{James Mason \& Nima Hejazi}\\ % Author(s)
\huge University of California, Berkeley\\ % University/organization
\end{minipage}
%
\hfill
%\begin{minipage}[b]{0.19\linewidth}
\includegraphics[scale=0.75]{logo_cal.jpg} % Logo or a photo of you, adjust its dimensions here
%\end{minipage}

\vspace{1cm} % A bit of extra whitespace between the header and poster content

%----------------------------------------------------------------------------------------

\begin{multicols}{3} % This is how many columns your poster will be broken into, a poster with many figures may benefit from less columns whereas a text-heavy poster benefits from more

%----------------------------------------------------------------------------------------
%	ABSTRACT
%----------------------------------------------------------------------------------------

\color{Navy} % Navy color for the abstract

\begin{abstract}

Sed fringilla tempus hendrerit. Vestibulum ante ipsum primis in faucibus orci luctus et ultrices posuere cubilia Curae; Etiam ut elit sit amet metus lobortis consequat sit amet in libero. Lorem ipsum dolor sit amet, consectetur adipiscing elit. Phasellus vel sem magna. Nunc at convallis urna. isus ante. Pellentesque condimentum dui. Etiam sagittis purus non tellus tempor volutpat. Donec et dui non massa tristique adipiscing. Quisque vestibulum eros eu. Phasellus imperdiet, tortor vitae congue bibendum, felis enim sagittis lorem, et volutpat ante orci sagittis mi. Morbi rutrum laoreet semper. Morbi accumsan enim nec tortor consectetur non commodo nisi sollicitudin.

\end{abstract}

%----------------------------------------------------------------------------------------
%	INTRODUCTION
%----------------------------------------------------------------------------------------

\color{SaddleBrown} % SaddleBrown color for the introduction

\section*{Introduction}

Aliquam non lacus dolor, \textit{a aliquam quam} \cite{Smith:2012qr}. Cum sociis natoque penatibus et magnis dis parturient montes, nascetur ridiculus mus. Nulla in nibh mauris. Donec vel ligula nisi, a lacinia arcu. Sed mi dui, malesuada vel consectetur et, egestas porta nisi. Sed eleifend pharetra dolor, et dapibus est vulputate eu. \textbf{Integer faucibus elementum felis vitae fringilla.} In hac habitasse platea dictumst.

Phasellus imperdiet, tortor vitae congue bibendum, felis enim sagittis lorem, et volutpat ante orci sagittis mi. Morbi rutrum laoreet semper. Morbi accumsan enim nec tortor consectetur non commodo nisi sollicitudin. Proin sollicitudin. Pellentesque eget orci eros. Fusce ultricies, tellus et pellentesque fringilla, ante massa luctus libero, quis tristique \textbf{purus urna nec nibh}.

%----------------------------------------------------------------------------------------
%	MATERIALS AND METHODS
%----------------------------------------------------------------------------------------

\section*{Data and Methodology}

Fusce magna risus, molestie ut porttitor in, consectetur sed mi. Vestibulum ante ipsum primis in faucibus orci luctus et ultrices posuere cubilia Curae; Pellentesque consectetur blandit pellentesque. Sed odio justo, viverra nec porttitor vel, lacinia a nunc. Suspendisse pulvinar euismod arcu, sit amet accumsan enim fermentum quis. In id mauris ut dui feugiat egestas. Vestibulum ac turpis lacinia nisl commodo sagittis eget sit amet sapien. Phasellus imperdiet, tortor vitae congue bibendum, felis enim sagittis lorem, et volutpat ante orci sagittis mi. Morbi rutrum laoreet semper. Morbi accumsan enim nec tortor consectetur non commodo nisi sollicitudin. Proin sollicitudin. Pellentesque eget orci eros. Fusce ultricies, tellus et pellentesque fringilla, ante massa luctus libero, quis tristique purus urna nec nibh. Proin sollicitudin. Pellentesque eget orci eros. Fusce ultricies, tellus et pellentesque fringilla, ante massa luctus libero, quis tristique purus.

%------------------------------------------------

\subsection*{Exploratory Analysis and Modeling}

Nulla vel nisl sed mauris auctor mollis non sed. 

\begin{equation}
E = mc^{2}
\label{eqn:Einstein}
\end{equation}

Curabitur mi sem, pulvinar quis aliquam rutrum. (1) edf (2)
, $\Omega=[-1,1]^3$, maecenas leo est, ornare at. $z=-1$ edf $z=1$ sed interdum felis dapibus sem. $x$ set $y$ ytruem. 
Turpis $j$ amet accumsan enim $y$-lacina; 
ref $k$-viverra nec porttitor $x$-lacina. 

Vestibulum ac diam a odio tempus congue. Vivamus id enim nisi:

\begin{eqnarray}
\cos\bar{\phi}_k Q_{j,k+1,t} + Q_{j,k+1,x}+\frac{\sin^2\bar{\phi}_k}{T\cos\bar{\phi}_k} Q_{j,k+1} &=&\nonumber\\ 
-\cos\phi_k Q_{j,k,t} + Q_{j,k,x}-\frac{\sin^2\phi_k}{T\cos\phi_k} Q_{j,k}\label{edgek}
\end{eqnarray}
and
\begin{eqnarray}
\cos\bar{\phi}_j Q_{j+1,k,t} + Q_{j+1,k,y}+\frac{\sin^2\bar{\phi}_j}{T\cos\bar{\phi}_j} Q_{j+1,k}&=&\nonumber \\
-\cos\phi_j Q_{j,k,t} + Q_{j,k,y}-\frac{\sin^2\phi_j}{T\cos\phi_j} Q_{j,k}.\label{edgej}
\end{eqnarray} 

$\mathbf{A}\underline{\xi}=\underline{\beta}$ Vim $\underline{\xi}$ enum nidi $3(P+2)^{2}$ lacina. Id feugain $\mathbf{A}$ nun quis; magno. Fusce convallis rutrum turpis, quis aliquet enim accumsan id. Vestibulum ullamcorper porttitor convallis. Integer sagittis interdum malesuada. Class aptent taciti sociosqu ad litora torquent per conubia nostra, per inceptos himenaeos. Sed adipiscing tristique orci at ullamcorper.
%----------------------------------------------------------------------------------------
%	RESULTS 
%----------------------------------------------------------------------------------------

\section*{Results}

Donec faucibus purus at tortor egestas eu fermentum dolor facilisis. Maecenas tempor dui eu neque fringilla rutrum. Mauris \emph{lobortis} nisl accumsan. Aenean vitae risus ante. Pellentesque condimentum dui. Etiam sagittis purus non tellus tempor volutpat. Donec et dui non massa tristique adipiscing.
%
\begin{center}\vspace{1cm}
\begin{tabular}{l l l l}
\toprule
\textbf{Treatments} & \textbf{Response 1} & \textbf{Response 2} \\
\midrule
Treatment 1 & 0.0003262 & 0.562 \\
Treatment 2 & 0.0015681 & 0.910 \\
Treatment 3 & 0.0009271 & 0.296 \\
\bottomrule
\end{tabular}
\captionof{table}{\color{Green} Table caption}
\end{center}\vspace{1cm}
%

Nulla ut porttitor enim. Suspendisse venenatis dui eget eros gravida tempor. Mauris feugiat elit et augue placerat ultrices. Morbi accumsan enim nec tortor consectetur non commodo. Pellentesque condimentum dui. Etiam sagittis purus non tellus tempor volutpat. Donec et dui non massa tristique adipiscing. Quisque vestibulum eros eu. Phasellus imperdiet, tortor vitae congue bibendum, felis enim sagittis lorem, et volutpat ante orci sagittis mi. Morbi rutrum laoreet semper. Morbi accumsan enim nec tortor consectetur non commodo nisi sollicitudin.

\begin{center}\vspace{1cm}
\includegraphics[width=0.8\linewidth]{placeholder}
\captionof{figure}{\color{Green} Subject Trajectories}
\end{center}\vspace{1cm}

In hac habitasse platea dictumst. Etiam placerat, risus ac.

Adipiscing lectus in magna blandit:

\begin{center}\vspace{1cm}
\begin{tabular}{l l l l}
\toprule
\textbf{Treatments} & \textbf{Response 1} & \textbf{Response 2} \\
\midrule
Treatment 1 & 0.0003262 & 0.562 \\
Treatment 2 & 0.0015681 & 0.910 \\
Treatment 3 & 0.0009271 & 0.296 \\
\bottomrule
\end{tabular}
\captionof{table}{\color{Green} Table caption}
\end{center}\vspace{1cm}

Vivamus sed nibh ac metus tristique tristique a vitae ante. Sed lobortis mi ut arcu fringilla et adipiscing ligula rutrum. Aenean turpis velit, placerat eget tincidunt nec, ornare in nisl. In placerat.

%----------------------------------------------------------------------------------------
%	CONCLUSIONS
%----------------------------------------------------------------------------------------

\color{SaddleBrown} % SaddleBrown color for the conclusions to make them stand out

\section*{Conclusions and Further Considerations}

\begin{itemize}
\item Pellentesque eget orci eros. Fusce ultricies, tellus et pellentesque fringilla, ante massa luctus libero, quis tristique purus urna nec nibh. Phasellus fermentum rutrum elementum. Nam quis justo lectus.
\item Vestibulum sem ante, hendrerit a gravida ac, blandit quis magna.
\item Donec sem metus, facilisis at condimentum eget, vehicula ut massa. Morbi consequat, diam sed convallis tincidunt, arcu nunc.
\item Nunc at convallis urna. isus ante. Pellentesque condimentum dui. Etiam sagittis purus non tellus tempor volutpat. Donec et dui non massa tristique adipiscing.
\end{itemize}

\color{DarkSlateGray} % Set the color back to DarkSlateGray for the rest of the content

\begin{center}\vspace{1cm}
\includegraphics[width=0.8\linewidth]{placeholder}
\captionof{figure}{\color{Green} Causal DAG}
\end{center}\vspace{1cm}

%----------------------------------------------------------------------------------------
%	ACKNOWLEDGEMENTS
%----------------------------------------------------------------------------------------

\section*{Acknowledgements}

Mention Yukie...

%----------------------------------------------------------------------------------------

\end{multicols}
\end{document}